\documentclass[12pt,a4paper]{article}
\usepackage{amsmath}
\usepackage{amsthm}
\usepackage{geometry}
\usepackage{enumitem}
\setitemize{noitemsep}
\usepackage{tabularx}
\usepackage{setspace}
\newcolumntype{x}{>{\centering\arraybackslash}X}

\newtheorem{lemma}{Lemma}

\usepackage{fontspec,xunicode}
\defaultfontfeatures{Mapping=tex-text}
\setsansfont{TeX Gyre Heros}
\setromanfont{Heuristica}

\usepackage{unicode-math}
\setmathfont{TeX Gyre Schola Math}

\usepackage[backend=biber,style=ext-authoryear-comp,natbib=true,maxbibnames=100,sorting=nyt,sortcites=true,giveninits=false,uniquename=false,uniquelist=false,isbn=false,doi=false,useprefix]{biblatex}

\addbibresource{grouprec.bib}

% removes "in:"
\renewbibmacro{in:}{%
  \ifentrytype{article}{}{\printtext{\bibstring{in}\intitlepunct}}}
% issue number in parentheses
\usepackage{xpatch}
% No dot before number of articles
\xpatchbibmacro{volume+number+eid}{%
  \setunit*{\adddot}%
}{%
}{}{}
% Number of articles in parentheses
\DeclareFieldFormat[article]{number}{\mkbibparens{#1}}

\DeclareFieldFormat*{title}{#1}
\DeclareFieldFormat{titlecase}{\MakeSentenceCase{#1}}
\DeclareFieldFormat[article]{titlecase:journaltitle}{\mkbibitalic{#1}}

% End of bibliography stuff

\usepackage{titlesec} %Package for adjusting the fonts of the \section and \subsection and so forth.
\titleformat{\section}{\Large\bfseries\sffamily}{\thesection.}{1em}{}
\titleformat{\subsection}{\large\bfseries\sffamily}{\thesubsection.}{1em}{}
\titleformat{\subsubsection}{\normalsize\bfseries\sffamily}{\thesubsubsection.}{1em}{}

\onehalfspacing

\begin{document}
\global\long\def\bpi{\bar{p}}%
\global\long\def\upi{\underline{p}}%



\section{Introduction}

Humans see other humans as members of groups, and they treat them differently
based on their group membership. Cognitively, they classify humans as
members of different groups, form stereotypes about other groups, and track 
group reputations. Motivationally, they form positive or negative attitudes to 
other groups and may discriminate against some groups. They may also reciprocate 
towards groups as a whole: that is, they may respond to one person's actions by
acting differently towards his or her fellow group members. We have experimental
evidence for group reciprocity, but so far there is no theoretical account of 
how it might have evolved. This paper sets out a theoretical model in which group
reciprocity can evolve. The requirements are quite minimal. Individuals know other
individuals' group membership, and they can recall outgroup members' past 
behaviour towards them. That is enough for group reciprocity to evolve in a 
repeated prisoner's dilemma. So, intergroup cognition alone is enough to make
intergroup reciprocity evolutionarily stable.

Group reciprocity is important in several ways. Theoretically, interethnic 
peace may be sustained by the threat of reciprocation. This could have allowed
humans to live in peace with neighbouring groups (perhaps punctuated by episodes of
punitive conflict), whereas chimpanzees always attack neighbouring bands on 
contact \citep{wrangham2012intergroup}. It could also enable deeper forms of intergroup cooperation, such as trade. So, the capacity for group reciprocity may be a stepping stone in the evolution of humans as a cooperative species. 

Group reciprocity entails that individuals keep mental records of groups'
past behaviour. These records could take the form of intergroup attitudes,
including stereotypes \citep{brewer1985psychology}. Thus, an evolutionary 
explanation for group reciprocity could also provide an explanation for why 
humans do not just categorize people into groups but also hold attitudes 
towards entire groups \citep{neuberg2008intergroup}.

Group reciprocity is also related to generalized upstream reciprocity 
or "paying it forward" \citep{boyd1989evolution,nowak2007upstream}. The 
literature has concluded that it is hard for 
generalized upstream reciprocity to evolve except in special circumstances. In
our model, a bounded form of generalized upstream reciprocity can evolve, 
because reciprocal cooperation between groups provides an evolutionary advantage
to those groups. Without groups, generalized upstream reciprocity does not allow
for this kind of group selection argument.

Practically, understanding the evolutionary basis
for intergroup categorization could help us to understand how it plays out 
in modern settings. That includes the prejudice and stereotyping mentioned above,
and also violent conflicts, including ethnic violence and civil war, where
intergroup reciprocity may be involved 
\citep{horowitz1985ethnicgroups,horowitz2001thedeadly,haushofer_both_2010}. 

Here is the logic behind our result. Groups of group-reciprocators
establish cooperative relationships with other groups of group-reciprocators,
while they defect against groups of selfish types. So, when some people are 
group reciprocators, individual fitness depends on one's group's reputation. Group 
reciprocators contribute to this reputation, which is a group-level public good. 
Thus being a group reciprocator is good for the group but costly for the individual. 
This then gives rise to a standard intrademic group selection model, where the group-level advantages balance out the individual-level costs \parencite{wilson1983group,wade1978critical}. Selfish free-riders within a group of mostly group-reciprocators get the highest payoff, but this is balanced
out because most selfish types are in mostly-selfish groups, who have too
few reciprocators to sustain intergroup cooperation.

We use computations to check the robustness of our theoretical model. Our
results confirm that even in small populations, or with relatively rare interactions,
group reciprocity can evolve.

Our model shows how group reciprocity can evolve, and it also has some insights
about how its form is shaped by evolutionary pressure. The most robust forms of
group reciprocity have high thresholds of cooperation - that is, reciprocators
only cooperate with groups a high proportion of whose members previously
cooperated towards them. These high thresholds make it less likely that selfish
free riders will invade, since only groups which are almost all reciprocators
get the benefit of mutual cooperation. The resulting equilibrium is 
"trigger happy": it sustains a high level of cooperation, but is also sensitive 
to small amounts of defection.

\section{Literature review}

Several experiments show evidence for group reciprocity, also known as "vicarious
revenge" \citep{lickel_vicarious_2006,gaertner2008whenrejection,stenstrom_roles_2008,hugh2017intergroup,hugh2019humans,romano2022direct}. There is also field 
evidence of reciprocity in violent intergroup conflict \citep{haushofer_both_2010}.
Anthropologists have recorded feud institutions, where group reciprocity is a
normative expectation and where there may be collective mechanisms for managing 
and adjudicating feuds \citep{boehm1984blood,chagnon1988lifehistories}. 

What we lack is a theory of how group reciprocity might evolve. While there
are evolutionary theories to explain individual reciprocity and revenge
\citep{mccullough2013cognitive}, there is none for group reciprocity. This is
a problem, because one can't simply assume that the same forces are behind the evolution of 
individual reciprocity and group reciprocity 
\citep{pietraszewski2013elementary,mccullough2013putting}. 
In particular, group reciprocity is a form of "upstream reciprocity", where an individual
who is helped or harmed by someone becomes more likely to respectively help 
or harm other third parties \citep{boyd1989evolution}. It is thought hard 
for upstream reciprocity to evolve,
because it does not target reciprocity in a way that might lead to stable 
bilateral relationships \citep{nowak2007upstream}.\footnote{By contrast,
we have satisfying theoretical explanations for why group membership might
matter for "downstream" or "indirect" reciprocity, where people help someone
who previously helped others. In particular, group membership may act as a 
"container" for this kind of reciprocity, because fellow group members
expect to interact often in future, or because coalitions are valuable
resources worth defending by third-party punishment
\citep{yamagishi2000thegroup,romano2022direct,delton2017psychology}.}

We show that the existence of groups makes this problem easier, by allowing 
different \emph{groups} to form bilateral relationships. This is true even 
though the groups don't possess any power of collective action (for instance, 
they can't collectively decide to sanction outgroups, or force their members 
to behave a certain way). Simply making individuals' group membership visible is 
enough to let group reciprocity evolve. Group reciprocity allows groups to 
enter mutually beneficial cooperative relationships, achieving a fitness 
advantage. This group selection logic does not apply when reciprocity is 
targeted at the whole population.

The logic of group reciprocity does not only explain group reciprocal motivations.
An essential part of any group reciprocal model is that individuals keep mental 
accounts of groups' previous behaviour, in some aggregate form. This mental
accounting provides an evolutionary explanation for why humans hold intergroup 
attitudes \citep{brewer1985psychology,kurzban2001can}. One simple descriptive 
framework is that groups, like individuals, are perceived on the two dimensions of
warmth and competence \citep{fiske2007universal}. Our model captures
the warmth dimension. It also shows why a group-level measurement can be cognitively
as important as individual-level characteristics: when humans are group-reciprocators,
aggregated group characteristics (how often that group has cooperated
with your group, and vice versa) predict individual behaviour better than individual
characeristics (how often you have cooperated with this individual).

\citet{fearon1996explaining} use a repeated-game framework, with 
two classes of possible equilibria, to explain how
different ethnic groups can live at peace. In their "spiral regime", defection
by any member of ethnic group A towards a member of B leads to subsequent 
defection by all members of B towards members of A for a fixed number of periods.
This is an infinitely repeated game with multiple equilibria. The goal is to explain
institutions which support interethnic cooperation; the spiral regime is analogous
to institutions like feuds. Our theory has a different setup and motivation. 
We examine the evolutionary stability of different types in a finitely-repeated game. 
In Fearon and Laitin, cooperation is supported by the threat of collective 
punishment. In our model, group reciprocity is evolutionarily stable because 
individual free-riding is balanced against group selection. So, our model
is designed to explain the evolution of "strong" reciprocal
motivations in humans \citep{gintis2000strong}, rather than the 
stability of institutions supporting interethnic peace. Ultimately, it is an 
empirical question whether interethnic peace and conflict are best explained 
by rational-actor models or by psychological models where "strong" reciprocity 
plays a role.

\section{Model}

We consider a mixed population of two types, selfish and group reciprocators (GR). At the beginning of each generation, the population randomly divides into a large number of groups of size~$G$ each.
Let~$p$ denote the population share of GR types. 
Let~$p_g$ be the proportion of GR in group~$g$, which is distributed binomially.

At every step~$t$, everybody interacts with everybody. In each pair, each individual chooses between cooperation and defection. Cooperation entails a cost~$c$ to the cooperator and a benefit~$b$ to her partner. Defection carries no costs. That is, each pair plays the following Prisoner's Dilemma game:
\begin{center}
    \begin{tabularx}{0.5\textwidth}{|x|x|x|}
        \hline
        &   Cooperate   &   Defect  \\
        \hline
        Cooperate   &   $b-c$   &   $-c$    \\
        \hline
        Defect  &   $b$ &   $0$   \\
        \hline
    \end{tabularx}
\end{center}

Selfish types always defect, A GR individual~$i$ starts by cooperating, and then cooperates with all individuals belonging to group~$g$ with a probability~$\phi(l_{gi})$, where~$l_{gi}$ is the proportion of individuals from group~$g$ who cooperated with individual~$i$ in round $t-1$.
$\phi(\cdot)$ is monotonically weakly increasing. We consider the cutoff strategy: 
$$
    \phi(l_{gi}) =
    \begin{cases}
        1   &   \text{if } l_{gi} \geq k  \\
        0   &   \text{otherwise.}
    \end{cases}
$$
The fitness is the payoff at the limit where~$t\to\infty$. Equivalently, since the game always settles to a stationary action profile, it is the average payoff of~$T$ rounds when~$T\to\infty$.

Given this behaviour, GR in all groups where $p_g \geq k$ help all individuals in other groups where $p_g \geq k$ and defect against members of all other groups. GR in other groups always defect.
%
Individuals' fitness therefore depends only on whether they are in a ``supraliminal" group with $p_g \geq k$, and on their type. Let~$q$ be the 
proportion of supraliminal groups. Let~$\bpi$ be the proportion of 
GR individuals in supraliminal groups (out of the total population in such groups). Let~$\upi$ be the proportion of GR individuals in subliminal groups (out of the total population in such groups). It follows that
\begin{itemize}
    \item Group reciprocators in supraliminal groups get a payoff of~$\bpi qb - qc$.
    \item Selfish types in supraliminal groups get~$\bpi qb$.
    \item Group reciprocators and selfish types in subliminal groups get~$0$.
\end{itemize}

After each generation, reproductive success is proportional to fitness, the total population size stays the same, and children are remixed randomly into new groups of the same size.
%
The mean fitness of the GR type is

\begin{equation*}
\frac{
  \bpi q(q(\bpi b - c))
}{
  p
}
\end{equation*}

and the mean fitness of selfish types is

\begin{equation*}
\frac{
  (1 - \bpi)q(q\bpi b)
}{
  1 - p
}
\end{equation*}

After rearranging, the mean fitness of reciprocators is higher if


\begin{equation}
\label{gr-wins}
\frac{ \bpi - p}{ 1 - p} \ge \frac{c}{b}.
\end{equation}

where 

\[
\bpi = E[p_g | p_g \ge k] = \frac{1}{G} \frac{\sum_{l=kG}^G l \mathrm{Binom}(l,G,p)}{\sum_{l=kG}^G \mathrm{Binom}(l,G,p)}
\]

The LHS of \eqref{gr-wins} is decreasing in $p$ and is equal
to the threshold $k$ when $p = 0$.\footnote{
    When the share of GR in the population is very small, the probability that $p_g = k$ conditional on~$p_g\geq k$ goes to one.
}

\begin{proof}
We will first prove an auxiliary lemma and then the main result.
\begin{lemma}
\label{lemma:eqiv_formulation}
Let $x_1,x_2,...,x_n$ be iid Binomially-distributed variables with probability of success $p$, and let $x_i=1$ denote the event where $x_i$ fails. Then showing that $\frac{ \bpi - p}{ 1 - p}$ decreases in $p$ is equivalent to showing that $\frac{p(x_1=1|S_n \leq k)}{p(x_1=1)}$ increases in $p$, where $S_n=x_1+x_2+...+x_n$.
\end{lemma}
\textbf{Proof of Lemma \ref{lemma:eqiv_formulation}}

First note that proving that the LHS of \eqref{gr-wins} is decreasing in $p$ is equivalent to proving that $1-\frac{ \bpi - p}{ 1 - p} = \frac{ 1- \bpi}{ 1 - p}$ is increasing in $p$. Second, note that 
% if $p$ is the probability of success of a Binomially-distributed variable $x_i$, then 
$1- \bpi$ captures the expected proportion of failures of a Binomially-distributed variable given that the proportion of successes was at least $k$, or, put differently, given that the proportion of failures was at most $k$. Finally, we can replace the expected proportion of failures with the probability of a failure. All in all, we get that proving that $\frac{ 1- \bpi}{ 1 - p}$ is increasing in $p$ is equivalent to showing that $\frac{p(x_1=1|S_n \leq k)}{p(x_1=1)}$ is increasing in $p$.
% hence decreasing in $q$.
\vspace{0.3cm}

\textbf{Proving the main result}
\vspace{0.3cm}

Lemma \ref{lemma:eqiv_formulation} implies that if $x_i=1$ denotes the event where $x_i$ fails, and $S_n$ counts the number of failures among $n$ trials, then we need to show that $\frac{p(x_1=1|S_n \leq k)}{p(x_1=1)}$ increases in the probability of success $p$.
For tractability, we will now revert to the more standard notation of $x_i=1$ as denoting the event where $x_i$ \textit{succeeds} (s.t. $S_n$ counts the number of successes among $n$ trials), and show that $\frac{p(x_1=1|S_n \leq k)}{p(x_1=1)}$ \textit{decreases} rather than increases in the probability of success $p$.
\vspace{0.6 cm}

\textbf{Proof}

\vspace{0.3 cm}
$\frac{p(x_1=1|S_n \leq k)}{p(x_1=1)}=\frac{p(s_n \leq k|x_1=1)}{p(s_n \leq k)}=\frac{p(s_{n-1} \leq k-1)}{qp(s_{n-1} \leq k)+pp(s_{n-1} \leq k-1)}$

\vspace{0.3cm}
$=\frac{p(s_{n-1} \leq k-1)}{q(p(s_{n-1} \leq k)-p(s_{n-1} \leq k-1))+p(s_{n-1} \leq k-1)}=\frac{p(s_{n-1} \leq k-1)}{qp(s_{n-1}=k)+p(s_{n-1} \leq k-1)}.$
\vspace{0.3cm}

Using a known result,\footnote{See equations (3) and (4) in https://mathworld.wolfram.com/BinomialDistribution.html.} according to which
\begin{equation*}
    p(s_n \leq k) = \frac{n!}{(n-k-1)!k!}\int_0^q t^{n-k-1}(1-t)^k dt = \frac{n!}{(n-k-1)!k!}\int_0^1 q^{n-k} s^{n-k-1} (1-qs)^k ds,
\end{equation*}
we get that $\frac{p(x_1=1|S_n \leq k)}{p(x_1=1)}$ is decreasing in $p$ if and only if $\frac{q \binom{n-1}{k} p^k q^{n-k-1}}{\frac{(n-1)!}{(n-k-1)!(k-1)!}\int_0^1 q^{n-k}s^{n-k-1}(1-qs)^{k-1} ds}$ is decreasing in $q$, i.e., if and only if $\frac{\binom{n-1}{k} p^k}{\frac{(n-1)!}{(n-k-1)!(k-1)!}\int_0^1 s^{n-k-1}(1-qs)^{k-1} ds}$ is decreasing in $q$.
\vspace{0.3cm}
Thus, it is sufficient to show that $\int_0^1 s^{n-k-1}(\frac{1-qs}{1-q})^{k-1}\frac{1}{1-q} ds$ is increasing in $q$. Since $(\frac{1-qs}{1-q})^{k-1}\frac{1}{1-q}$ is non-decreasing in $q$ for every $s \in [0,1]$, the proof is complete.

\end{proof}
%
It follows that there is a unique ESS with a positive share of group
reciprocators if and only if $k > \frac{c}{b}$. Otherwise the population is homogeneously selfish in the unique ESS.

\end{document}
%%% THE REST OF THE ATTEMPTS TO PROVE ARE IN THE SEPARATE FILE

\subsection*{Proof that \eqref{gr-wins} goes to $k$ as $p \to 0$}


Consider

\[
\frac{\bpi - p}{1-p}
\]
The limit as $p \to 0$ is given by $\bpi$. For $k=0$ this is
0 since $\bpi = p$. For $k>0$, write
\[
\bpi=\frac{1}{G}\frac{\sum_{l=kG}^{G}l\phi(l,G,p)}{\sum_{l=kG}^{G}\phi(l,G,p)}=\frac{1}{G}\frac{\sum_{l=kG}^{G}l\binom{G}{l}p^{l}(1-p)^{G-l}}{\sum_{l=kG}^{G}\binom{G}{l}p^{l}(1-p)^{G-l}}.
\]

The limit of this is given by L'H\^{o}pital's rule, since the denominator
goes to zero. Differentiating top and bottom gives
\[
\frac{1}{G}\frac{\sum_{l=kG}^{G}l\binom{G}{l}(l-Gp)p^{l-1}(1-p)^{G-l-1}}{\sum_{l=kG}^{G}\binom{G}{l}(l-Gp)p^{l-1}(1-p)^{G-l-1}}
\]
so we have to apply the rule again. Continuing thus we eventually hit:
\[
\frac{1}{G}\frac{kG\binom{G}{l}(l-Gp)(l-1-Gp)(\cdots)(l-kG-Gp)p^{0}(1-p)^{G-2kG}+\cdots}{\binom{G}{l}(l-Gp)(l-1-Gp)(\cdots)(l-kG-Gp)p^{0}(1-p)^{G-2kG}+\cdots}
\]
 where all but the first terms vanish, leaving
\[
\frac{1}{G}\frac{kG\binom{G}{l}(l-Gp)(l-1-Gp)(\cdots)(l-kG-Gp)p^{0}(1-p)^{G-2kG}}{\binom{G}{l}(l-Gp)(l-1-Gp)(\cdots)(l-kG-Gp)p^{0}(1-p)^{G-2kG}}=\frac{1}{G}kG=k.
\]


(Old stuff)

\begin{align}
    q & = & 1 - CDF(G, p') \\
    \bpi & = & E[p_k | p_k \ge c] \\
    \upi & = & E[p_k | p_k < c]
\end{align}

Where $CDF$ is the binomial distribution of $p'$ out of $G$ successes, and
the conditional expectations are also taken according to this distribution.

Computations are in continuous-model-computations.R in dropbox. Plotting
reveals:

When $p$ is sufficiently below $c$, $p' = p$ and group reciprocity is 
selectively neutral, since group reciprocators and selfish types are
almost all in subliminal groups.

For some parameters, and intermediate values of $p$, $p'>p$ so that
group reciprocity is being selected for. That is, some groups successfully
manage to cooperate with other groups and have higher average fitness.

For high enough values of $p$, $p' < p$. This is because when almost everyone
is cooperating, a selfish type is almost surely in a supraliminal group
and benefits by freeriding.

Conjectures and questions.

1. What if the threshold can evolve? Might higher thresholds always 
be beneficial? If so, probably only few group reciprocators exist,
and very few are in supraliminal groups. One way to do this is
generalize the model to have high- and low-threshold types, and see
who wins. Note that in the model as is, a selfish type just has a
"threshold" of above 1. Another approach is to ask if a single
person with a different threshold can invade. He will change the
behaviour of other groups to him, but as there are many groups,
we can ignore the effects on other people's fitness and just
ask if he outcompetes other group reciprocators.
\end{document}
