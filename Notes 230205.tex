\documentclass[12pt,a4paper]{article}
\usepackage{amsmath}
\usepackage{geometry}
\usepackage{enumitem}
\setitemize{noitemsep}
\usepackage{tabularx}
\usepackage{setspace}
\newcolumntype{x}{>{\centering\arraybackslash}X}

\usepackage{fontspec,xunicode}
\defaultfontfeatures{Mapping=tex-text}
\setsansfont{TeX Gyre Heros}
\setromanfont{Heuristica}

\usepackage{unicode-math}
\setmathfont{TeX Gyre Schola Math}

\usepackage{titlesec} %Package for adjusting the fonts of the \section and \subsection and so forth.
\titleformat{\section}{\Large\bfseries\sffamily}{\thesection.}{1em}{}
\titleformat{\subsection}{\large\bfseries\sffamily}{\thesubsection.}{1em}{}
\titleformat{\subsubsection}{\normalsize\bfseries\sffamily}{\thesubsubsection.}{1em}{}

\onehalfspacing

\begin{document}
\global\long\def\bpi{\bar{p}}%
\global\long\def\upi{\underline{p}}%


\section*{Notes}

\begin{itemize}
	\item Getting from intergroup cognition to intergroup motivation. Reference Tajfel or Turner.
	\item Fearon and Laitin: they have a folk theorem setup where cooperation is sustained by fear of punishment. We have group reciprocity.
	\item IGS models.
	\item We want to explain interethnic cooperation. Similar to Fearon and Laitin, but different mechanism.
\end{itemize}

\section*{Structure}

\begin{itemize}
	\item Motivation: intergroup motivations from intergroup cognition. Tajfel and/or Turner.
	\item We also contribute to other literatures:
	\begin{itemize}
		\item Evolution of cooperation, indirect reciprocity, Boyd and Richerson upstream/downstream.
		\item Interethnic cooperation, Fearon and Laitin.
	\end{itemize}
	\item Rationale of the model, highlight key insight.
	\item Model.
	\item Simulations to extend the model?
	\item Conclusion.
\end{itemize}

\end{document}